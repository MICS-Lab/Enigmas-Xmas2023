\documentclass[a4paper, top=10mm]{article}
%for writing from the top
\usepackage{fullpage}
%for math
\usepackage{amsmath}
\usepackage{mathrsfs}
\usepackage{amsthm}
%for images
\usepackage{graphicx}
%for color
\usepackage{xcolor}
%for title
\title{\textbf{\huge{Title}}}
\author{Enigma n\textsuperscript{o}3}
\date{14\textsuperscript{th} December 2023}

\newtheorem*{hint}{Hint}

\addtolength{\voffset}{-2cm}
\addtolength{\textheight}{5cm}


\begin{document}
	\maketitle
	
	Paul-Henry Cournède (PHC) and Céline Hudelot (CH) are sitting face to face.
	We tell them there are $50$ black chocolates, and $50$ white chocolates, for a total of $100$ chocolates.
	There are $50$ chocolates behind Céline, and $49$ behind Paul-Henry.
	One chocolate is hidden.
	
	The two professors can not see the chocolates behind them-self, but they can see the ones behind the other professor.
	We ask them iteratively "Do you know how many black / white chocolates there are behind you?".
	
	They answer as follows:
	
	\begin{enumerate}
		\item \textbf{PHC}: "I do not know."
		\item \textbf{CH}: "I do not know."
		\item \textbf{PHC}: "I do not know."
		\item \textbf{CH}: "I do not know."
		\item \textbf{PHC}: "I do not know."
		\item \textbf{CH}: "I do not know."
		\item \textbf{PHC}: "I do not know."
		\item \textbf{CH}: "I do not know."
		\item \textbf{PHC}: "I do not know."
		\item \textbf{CH}: "I do not know."
		\item \textbf{PHC}: "I do not know."
		\item \textbf{CH}: "I do not know."
		\item \textbf{PHC}: "I do not know."
		\item \textbf{CH}: "I do not know."
		\item \textbf{PHC}: "I know!"
		\item \textbf{CH}: "I know!"
	\end{enumerate}
	
	\begin{center}
		\includegraphics[height=150pt]{00example.png}\\
		Legend
	\end{center}
	
	[question...]
	
\end{document}